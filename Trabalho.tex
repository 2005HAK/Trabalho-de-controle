\documentclass[12pt, a4paper]{article}

\usepackage[utf8]{inputenc}
\usepackage[T1]{fontenc}
\usepackage[brazil]{babel}
\usepackage{geometry}
\usepackage{amsfonts, amsmath, amssymb}
\usepackage{graphicx}
\usepackage{hyperref}
\usepackage{titlesec}
\usepackage{indentfirst}
\usepackage{float}

\geometry{a4paper, margin={3cm,2cm,2cm,3cm}}
\setlength{\parindent}{1.25cm}

% --- Configuração de Títulos (com titlesec) ---
% Força os títulos a terem 12pt (\normalsize) e negrito

\titleformat{\section}
  {\normalsize\bfseries\MakeUppercase} % formato: 12pt e negrito
  {\thesection.}         % rótulo (número): "1."
  {1em}                  % separação: 1 "M" de espaço
  {}                     % código antes do título (pode deixar vazio)

\titleformat{\subsection}
  {\normalsize\MakeUppercase} % formato: 12pt e negrito
  {\thesubsection.}      % rótulo (número): "1.1."
  {1em}                  % separação
  {}                     % código antes

\titleformat{\subsubsection}
  {\normalsize\bfseries} % formato: 12pt e negrito
  {\thesubsubsection.}   % rótulo (número): "1.1.1."
  {1em}                  % separação
  {}                     % código antes

\newcommand{\universidade}{Universidade Federal de Santa Catarina}
\newcommand{\centro}{Centro Tecnológico de Joinville - CTJ}
\newcommand{\disciplina}{EMB5641 - Sistemas de Controle}
\newcommand{\curso}{Engenharia Mecatrônica}
\newcommand{\professor}{Prof. Alexandro Garro Brito}
\newcommand{\titulo}{Controle de profundidade de veículo subaquático autônomo (AUV)}
\newcommand{\autor}{Hebert Alan Kubis}
\newcommand{\matricula}{23102547}

\newcommand{\vehicleMass}{M_t}
\newcommand{\vehicleVolume}{V_t}
\newcommand{\SensorPressH}{h_p}

\begin{document}

\begin{titlepage}
    \begin{center}
        {\bfseries \MakeUppercase{\universidade}}\\
        {\bfseries \MakeUppercase{\curso}}\\[3cm]
        
        {\MakeUppercase{\autor}}\\

        \vfill
        {\bfseries \MakeUppercase{\titulo}}\\
        \vfill

        {Joinville}\\
        {\number\year}

    \end{center}
\end{titlepage}
 
% Folha de rosto (ABNT)
\begin{titlepage}
    \begin{center}
        {\MakeUppercase{\autor}}\\
        \vfill

        {\bfseries \MakeUppercase{\titulo}}\\[2cm]
        
        \hspace*{8cm}
        \parbox[t]{\dimexpr\textwidth-8cm\relax}{
            {Trabalho apresentado ao \centro\ da\ \universidade\ como requisito 
            parcial para a disciplina \disciplina.}\\[1cm]
            {\professor}\\
        }\\

        \vfill

        {Joinville}\\
        {\number\year}

    \end{center}
\end{titlepage}

\begin{abstract}
    Ainda tenho que escrever o trabalho para poder resumir
\end{abstract}
\clearpage

\tableofcontents
\clearpage

\section{Introdução}
    No presente relatório será descrito o processo de modelamento do sistema de controle de 
    profundidade de um AUV (Veículo Autônomo Submarino, do inglês Autonomous Underwater Vehicle). 
    O processo de modelamento desse sistema se dá pela união entre duas necessidades, uma é a 
    realização do trabalho avaliativo de Sistemas de controle e a outra é a necessidade de um 
    controle de profundidade preciso para o veículo Yvy da equipe Terra Competition da \universidade, campus Joinville.
\clearpage

\section{Definição do sistema}
    O sistema a ser controlado é um AUV (Autonomous Underwater Vehicle) que utiliza 6 motores, 
    4 para controle no plano xOy e 2 para controle no eixo Z (vertical). Outra característica é 
    que este veículo possui 2 cilindros em seu interior. Um deles está posicionado encostado na 
    parte de cima ao longo do eixo X, sendo que neste fica toda a eletrônica responsável por 
    controlar o veículo, e um cilindro menor na mesma orientação encostado na parte inferior do 
    veículo, sendo que esse é o cilindro responsável por abrigar a bateria.
    
    Dado essa descrição do veículo, pode-se definir o objetivo do sistema de controle: 
    {\bfseries estabilizar o veículo em uma dada profundidade.}

    \subsection{Considerações}
        Para o modelamento do sistema serão feitas as seguintes considerações:
        \begin{itemize}
            \item Será avaliado somente o movimento vertical do veículo, assumindo que o peso dele está homogeneamente distribuído ao longo dos outros eixos;
            \item Será assumido que os dois motores que controlam o movimento vertical estão equidistantes ao longo do eixo Y em relação à origem e estão sobre a origem em X. Desta forma, não será considerado o torque gerado pelos motores superiores sobre os eixos X e Y caso estes estejam desalinhados;
            \item O eixo Z positivo é direcionado para baixo no veículo;
            \item O veículo está completamente submerso.
        \end{itemize}

    \subsection{Fontes de dados}
        Para controlar o veículo são necessárias fontes de dados para saber como atuar na malha. Com isso, os sensores disponíveis para analisar o comportamento do veículo são:
        \begin{itemize}
            \item Sensor de profundidade, fonte precisa de dados de pressão;
            \item Pixhawk PX4, fonte de dados precisa de aceleração;
            \item Cruzamento dos dados desses dois sensores, por integração e derivação numérica, para obter a velocidade.
        \end{itemize}

\section{Modelamento do sistema}
    \subsection{Variaveis do sistema}
        Para modelar o sistema de controle de profundidade do AUV alguns dados precisam ser 
        considerados, sendo que muitos, e talvez todos, podem sofrer alterações. Para modelar 
        um sistema que permita controlar o veículo mesmo com a alteração de suas características, 
        será consideradas variáveis no lugar dos dados do veículo. O seguintes dados serão usados:
        
        \begin{itemize}
            \item {$\vehicleMass$} - Massa do veículo;
            \item {$\vehicleVolume$} - Volume do veículo;
            \item {$\SensorPressH$} - Distância que o sensor de pressão está do topo do AUV;
        \end{itemize}
    
    \subsection{Equilibrio de forças}
        Para modelar o sistema de controle de profundidade do AUV, primeiramente é necessário 
        analisar as forças que atuam sobre o veículo. As forças que atuam sobre o veículo são:
        
        \begin{itemize}
            \item Força peso, $F_{p}$;
            \item Força de empuxo, $F_{e}$;
            \item Força gerada pelos motores, $F_{m}$;
            \item Força de arrasto, $F_{d}$.
        \end{itemize}

        Fazendo o somatorio das forças que atuam sobre o veículo e aplicando a segunda Lei 
        de Newton tem-se:

        \begin{equation}
            \sum F = F_{p} - F_{e} + F_{m} - F_{d} = m \cdot a
        \end{equation}

        Onde $a$ é a aceleração do veículo ao longo do eixo Z, ou seja, $a = \frac{d^2z}{dt^2}$.

        Substituindo as forças na equação acima:

        \begin{equation}
            M_t \cdot \frac{d^2z}{dt^2} = M_t \cdot g - \rho \cdot g \cdot V_t + F_{m} - B_d \cdot \frac{dz}{dt}
            \label{eq:somatorio de forcas}
        \end{equation}

        Assumindo que o AUV se encontra numa posição que é seu ponto de equilibrio $z_0$ mais uma 
        perturbação $\Delta z$, ou seja, $z = z_0 + \Delta z$. Com isso, a equação \ref{eq:somatorio de forcas} fica:

        \begin{equation}
            M_t \cdot \frac{d^2(z_0 + \Delta z)}{dt^2} = (M_t \cdot g - \rho \cdot g \cdot V_t) + F_{me} + \Delta F_{m} - B_d \cdot \frac{d(z_0 + 
            \Delta z)}{dt}
        \end{equation}

        Onde $F_{me}$ é a força dos motores no ponto de equilibrio e $\Delta F_{m}$ é a variação 
        dessa força. Sabendo que a derivada de uma constante é nula, ou seja, $\frac{dz_0}{dt} = 0$ e 
        $\frac{d^2z_0}{dt^2} = 0$, tem-se:

        \begin{equation}
            M_t \cdot \frac{d^2 \Delta z}{dt^2} = (M_t \cdot g - \rho \cdot g \cdot V_t) + F_{me} + \Delta F_{m} - B_d \cdot \frac{d \Delta z}{dt}
            \label{eq:somatorio de forcas com delta}
        \end{equation}

        Da equação \ref{eq:somatorio de forcas} sabe-se que no ponto de equilibrio a parcela 
        $\frac{d^2z}{dt^2} = 0$ e $\frac{dz}{dt} = 0$, logo:

        \begin{equation}
            0 = M_t \cdot g - \rho \cdot g \cdot V_t + F_{me}
        \end{equation}

        \begin{equation}
            F_{me} = \rho \cdot g \cdot V_t - M_t \cdot g
            \label{eq:forca de equilibrio dos motores}
        \end{equation}

        Substituindo a equação \ref{eq:forca de equilibrio dos motores} na equação 
        \ref{eq:somatorio de forcas com delta} tem-se:

        \begin{equation}
            M_t \cdot \frac{d^2 \Delta z}{dt^2} = \Delta F_{m} - B_d \cdot \frac{d \Delta z}{dt}
        \end{equation}

        Isolando os termos:

        \begin{equation}
            M_t \cdot \frac{d^2 \Delta z}{dt^2} + B_d \cdot \frac{d \Delta z}{dt} = \Delta F_{m}
        \end{equation}

        Agora, assumindo condições iniciais nulas, ou seja, $\Delta z(0) = 0$ e $\frac{d \Delta z(0)}{dt} = 0$, 
        aplicando a Transformada de Laplace na equação acima tem-se:

        \begin{equation}
            M_t \cdot s^2 \cdot \Delta Z(s) + B_d \cdot s \cdot \Delta Z(s) = \Delta F_{m}(s)
        \end{equation}

        Reorganizando a equação para obter a função de transferência:

        \begin{equation}
            \frac{\Delta Z(s)}{\Delta F_{m}(s)} = \frac{1}{M_t \cdot s^2 + B_d \cdot s}
        \end{equation}

        Como o sistema encontrado é de segunda ordem, divide-se o numerador e o denominador por 
        $M_t$ para obter a forma padrão da função de transferência:

        \begin{equation}
            \frac{\Delta Z(s)}{\Delta F_{m}(s)} = \frac{\frac{1}{M_t}}{s^2 + \frac{B_d}{M_t} \cdot s}
        \end{equation}

        Com isto é possivel analisar que o sistema encontrado é de 2ª ordem do tipo 1, pois o maior 
        grau do denominador é 2 e, reorganizando os termor, é possivel visualizar que possui um polo 
        na origem, que torna o erro em estado estacionario nulo para uma entrada degrau.

        Por conta da natureza do sistema a ser controlado percebe-se que que o controlador a ser 
        implementado deve ser um P ou PD caso seja necessária uma resposta mais rápida.
    
    \subsection{Parametros do sistema}
        Para fazer a análise do sistema é necessario definir os parametros que o compõem, e como este é um veiculo real os parametros dele são os seguintes:

        \begin{itemize}
            \item Massa do veículo, {$\vehicleMass$} = 19 kg;
            \item Volume do veículo, {$\vehicleVolume$} =  0.014 m$^3$;
            \item Coeficiente de arrasto, $B_d$ = 42 kg/s;
            \item Densidade da água de piscina, $\rho$ = 1000 kg/m$^3$;
            \item Aceleração da gravidade, $g$ = 9.81 m/s$^2$.
        \end{itemize}

        A massa do veiculo foi medida com uma balança no Laboratório de Interação Fluido-Estrutura (LIFE) da \universidade. Ja o volume foi obtido a partir do modelo 
        CAD do veiculo, que foi feito no SolidWorks.

        Para obter o coeficiente de arrasto foi assumido que o AUV é um cubo de lado 0.4 m ja que suas dimensões são de 0.38 x 0.4 x 0.38 m. Assim, calculando o número de Reynolds para $\rho$ = 
        1000 kg/m$^3$, $\mu$ = 0.001 Pa.s, $v$ = 0.1 m/s e $L$ = 0.4 m tem-se: 

        \begin{equation}
            Re = \frac{\rho \cdot v \cdot L}{\mu} = \frac{1000 \cdot 0.1 \cdot 0.4}{0.001} = 40000
        \end{equation}

        Com este valor que é Re > $10^4$ o coeficiente de arrasto para um cubo é $C_d = 1.05$ \cite{Cd}. Mas, como é possivel ver na 
        equação \ref{eq:somatorio de forcas} não é de $C_D$ que se necessita, mas sim do coeficiente $B_d$ que é usado para tratar o 
        arrasto de forma linear. Assim, para obter $B_d$ é necessario definir uma velocidade de operação do veiculo, que foi assumida 
        como 0.5 m/s para os calculos se baseando em velocidades de operação de outros AUVs, como o BlueROV2 da BlueRobotics 
        \cite{bbauv4.5} e o BBAUV 4.5 da Bumblebee \cite{bluev2} que podem se mover à até 1.5 m/s.

        Com isso, o coeficiente $B_d$ é obtido a partir da formula do arrasto:

        \begin{equation}
            F_d = \frac{1}{2} \cdot \rho \cdot v_{op}^2 \cdot C_d \cdot A
        \end{equation}

        Onde $A$ é a área de seção transversal do veiculo, que é 0.16 m$^2$ (0.4 x 0.4 m), $v_{op}$ é a velocidade de operação (0.5 m/s), $\rho$ é a densidade do 
        fluido (1000 kg/m$^3$) e $C_d$ é o coeficiente de arrasto (1.05). Assim, unindo isto a formula do arrasto linear:

        \begin{equation}
            F_d = B_d \cdot v_{op} = \frac{1}{2} \cdot \rho \cdot v_{op}^2 \cdot C_d \cdot A
        \end{equation}

        Isolando $B_d$ tem-se:

        \begin{equation}
            B_d = \frac{\frac{1}{2} \cdot \rho \cdot v_{op}^2 \cdot C_d \cdot A}{v_{op}} = \frac{1}{2} \cdot \rho \cdot v_{op} \cdot C_d \cdot A
        \end{equation}

        Substituindo os valores conhecidos:

        \begin{equation}
            B_d = \frac{1}{2} \cdot 1000 \cdot 0.5 \cdot 1.05 \cdot 0.16 = 42 \,\, kg/s
        \end{equation}

        É importante ressaltar que o coeficiente de arrasto $B_d$ assumido é válido em torno da velocidade de operação escolhida, e que uma velocidade maior
        fara com que o controlador seja atue de maneira lenta no sistema, ja que o arrasto será maior do que previsto. Ja se a velocidade for menor 
        o controlador atuará de maneira mais agressiva, podendo causar oscilações no sistema.

    \subsection{Análise do sistema}
        Com os parametros definidos, a função de transferência do sistema fica:

        \begin{equation}
            \frac{\Delta Z(s)}{\Delta F_{m}(s)} = \frac{\frac{1}{19}}{s^2 + \frac{42}{19} \cdot s} = \frac{0.0526}{s^2 + 2.2105 \cdot s}
        \end{equation}

        Onde os valores foram arredondados para 4 casas decimais. 

        Com isso é possivel fazer a análise do sistema plotando o seu lugar das raízes no matlab, como mostrado na figura \ref{fig:root locus Gs}.

        \begin{figure}[H]
            \centering
            \includegraphics[width=0.8\textwidth]{img/root_locus_Gs.jpg}
            \caption{Lugar das raízes do sistema}
            \label{fig:root locus Gs}
        \end{figure}

        Com isso é possivel perceber que o sistema em malha aberta é marginalmente estável por conta do polo em zero ja mencionado. Outro ponto importante 
        é que um controlador proporcional é suficiente para estabilizar o sistema, ja que os polos de malha fechada nunca cruzam para o semiplano direito, 
        portanto isso confirma a análise anterior de que um controlador P ou PD é suficiente para controlar o sistema.

\section{Controlador}
    \subsection{Requisitos de desempenho}
        Para o sitema de controle de profundidade do AUV é necessario definir os requisitos de desempenho do sistema controlado. Para isso, é necessário analisar 
        o ambiente e as condições em que o AUV opera. Pensando que o AUV pode estar a 1m de profundidade e pode ser dado um comando em degrau para que ele se posicione 
        a 5m de profundidade, um percentual de sobressinal aceitavel seria algo como $PSS = 5\%$. A principal justificativa para limitar o Percentual de Sobressinal 
        (PSS) é a \textbf{segurança operacional e a integridade do veículo}.

        O AUV Yvy pode ser comandado a operar em profundidades muito próximas do solo (seja o fundo do mar ou o piso do tanque de testes). Por exemplo, se o sistema 
        de controle comandar uma profundidade de referência $z_{ref} = 5,0 \text{ m}$ quando o solo está mapeado em $z_{solo} = 5,1 \text{ m}$, um PSS de apenas 5\% 
        já seria catastrófico. O cálculo do sobressinal seria: 
        
        \begin{equation}
            \Delta z = z_{ref} - z_{inicial} = 5,0 \text{ m} - 1,0 \text{ m} = 4,0 \text{ m}
        \end{equation}

        \begin{equation}
            z_{pico} = z_{ref} + (\Delta z \times \frac{PSS}{100}) = 5,0 \text{ m} + (4,0 \text{ m} \times 0,05) = 5,2 \text{ m}
        \end{equation}

        Neste cenário, o AUV tentaria atingir 5,2 m, colidindo com o solo em 5,1 m. Mesmo que a velocidade no pico seja momentaneamente nula, a colisão poderia 
        danificar sensores frágeis (como câmeras ou sonar) e levantar sedimentos, comprometendo a missão.

        Portanto, um $PSS \le 5\%$ é definido como um requisito de segurança crítico para garantir uma margem operacional mínima, aliado a uma condição de que 
        $z_{ref} \le 0.95 \cdot h_{max}$, onde $h_{max}$ é a profundidade máxima do ambiente onde o AUV se encontra.

        Outro critério que deve ser definido é o tempo de acomodação $T_s$ do sistema controlado, que é o tempo entre dar a entrada no sistema e ele realmente chegar à 
        profundidade especificada. Pensando em uma diferença de profundidade de 4m, partindo de 1m e chegando em 5m, e a velocidade de operação definida como 0.5m/s, é aceitavel 
        definir um tempo de acomodação de 10s, visto que demoraria cerca de 8s para que o veiculo chegue a profundidade especificada e mais 2s de espera definido arbitrariamente
        para que ele estabilize.

        O ultimo requisito de desempenho que o sistema controlado deve atingir é que o erro em regime permanente seja 0, ou seja, que o AUV realmente chegue na referencia e 
        fique nela. De todo modo, este requisito é garantido pelo que ja foi discutido antes, o polo na origem, que se traduz como um integrador, classificando o sistema como de 
        2ª ordem do tipo 1, ou seja, com $e_{ee}$ nulo para uma entrada em degrau.

    \subsection{Análise do lugar das raízes}
        Na imagem \ref{fig:root locus Gs} foi possivel visualizar a estabilidade da malha para diferentes valores de $K_p$. Agora, com os requisitos de desempenho estabeleciddos 
        é possível definir as regioes no lugar das raizes que atendem aos requisitos. Primeiramente deve-se estabelecer a equação da malha fechada do sistema, sendo esta a seguinte ao 
        assumir a utilização de somente um controlador $K_p$:

        \begin{equation}
            \frac{\Delta Z(s)}{\Delta F_m(s)} = \frac{0.0526 K_p}{s^2 + 2.2105s + 0.0526 K_p}
        \end{equation}

        Com isso sabe-se que:

        \begin{equation}
            \omega_n^2 = 0.0526K_p
        \end{equation}

        e que:

        \begin{equation}
            2 \zeta \omega_n = 2.2105
            \label{eq:zetaomegan}
        \end{equation}

        Para o requisito de tempo de subida tem-se que:

        \begin{equation}
            T_s \approx \frac{4}{|\zeta \omega_n|} 
            \label{eq:timerise}
        \end{equation}

        Substituindo a equação \ref{eq:zetaomegan} em \ref{eq:timerise}:

        \begin{equation}
            T_s \approx \frac{4}{|\zeta \omega_n|} < 10
        \end{equation}

        \begin{equation}
            |\zeta \omega_n| < 0.4
        \end{equation}

        Da planta sabe-se que $\zeta \omega_n$ é a parte real dos polos e que este termo é negativo, logo:

        \begin{equation}
            \zeta \omega_n < -0.4
        \end{equation}

        Outro ponto é 

    \subsection{Projeto do controlador}

    \subsection{Resposta em frequência}

\section{Implementação}
    \subsection{Discretização do controlador}

    \subsection{Aplicação no sistema}

    \subsection{Testes e resultados}

\section{Conclusão}

\bibliographystyle{plain}
\bibliography{referencias}

\end{document}